\documentclass{spaceA}
\author{d8ahaze}
\class{2::4.calculus, pid06\_fB}
\date{Start: 01.03.22; Last: \today; Finish: dd.mm.yy}

\usepackage[T2A]{fontenc} % inclusion cyrillic encodings
\usepackage[russian,english]{babel}
\usepackage[default]{comfortaa}
\usepackage{amsmath}
%\usepackage{esdiff}
%\usepackage{cleveref}

\newcommand*{\ruenc}{\fontencoding{T2A}\selectfont} % ruenc: command for Russian encoding enablement
\newcommand*{\latenc}{\fontencoding{T1}\selectfont} % latenc: command for Latin encoding enablement

\title{Введение: A.NOTES} % Chapter 1

\graphicspath{{./frame/bin/}}

\begin{document} \maketitle

\asec

\section{About}

\begin{itemize}
  \item Tool 1: Natural Numbers
    Overview: content

    Path: ~/1.1/2::1.main1/pid01\_fB
  \item Tool N: Subset
    Overview: content

    Path: ~/1.1/1::2.main1/pid01\_fB
  \item ...
  \item Tool N: Bijection
    Overview: content

    Path1: ~/1.1/1::2.main1/pid zxc

    Path2: ~/1.1/2::4.main1/pid01\_fB/ch1/asecX/cellN
\end{itemize}

\section{Comprehend}

\begin{enumerate}
  \item - Cell: $\mathbb{N}$ -- множество натуральных чисел (tool: 1).
  \item - Cell: $\mathbb{Z}$ -- множество целых чисел.
  \item - Cell: $\mathbb{Q}$ -- множество рациональных чисел.
  \item - Cell: $\mathbb{I}$ -- множество иррациональных чисел.
  \item - Cell: $\mathbb{R}$ -- множество действительных чисел.
  \item - Cell: $\mathbb{C}$ -- множество комплексных чисел.
  \item - Cell: $x$ -- элемент множества $A$: $x \in A$.
  \item - Cell: $x$ -- не является элементом множества $A$: $x \notin A$.
  \item - Cell: $\forall x \left ( x \in A \Rightarrow x \in B \right ) = A \subset B$ (tool: N).
  \item - Cell: If $A \subset B$ and $B \subset A$, so $A = B$.
  \item - Cell: $\varnothing = \{ \}$, $\forall A: \varnothing \in A$.
\end{enumerate}

\section{Analysis of Set}

\begin{itemize}
  \item Natural Numbers are abstract line of squares
  \item Each square is corresponding to some number aside from zero.
\end{itemize}

\section{Set of Cells: Опреации над множествами}

\begin{enumerate}
  \item - Cell: $A \cup B$ or $A + B := \left (x \in A \right ) \lor \left (x \in B \right )$.
  \item - Cell: $A \cap B$ or $A \cdot B := \left (x \in A \right ) \land \left (x \in B \right )$
  \item - Cell: $A \cap B = \varnothing$ -- множествa $A$ и $B$ не пересекаются.
  \item - Cell: $\mathbb{I}$ -- множество иррациональных чисел.
  \item - Cell: $\mathbb{R}$ -- множество действительных чисел.
  \item - Cell: $\mathbb{C}$ -- множество комплексных чисел.
  \item - Cell: $x$ -- элемент множества $A$: $x \in A$.
  \item - Cell: $x$ -- не является элементом множества $A$: $x \notin A$.
  \item - Cell: $\forall x \left ( x \in A \Rightarrow x \in B \right ) = A \subset B$.
  \item - Cell: If $A \subset B$ and $B \subset A$, so $A = B$.
  \item - Cell: $\varnothing = \{ \}$, $\forall A: \varnothing \in A$.
\end{enumerate}

\section{Set of Cells: Эквивалентные и неэквивалентные множества}

\begin{enumerate}
  \item - Cell: Взаимно однозначное соответствие -- каждому элементу множества $A$ сопоставлен один и только один элемент множества $B$, так что различным элементам множества $A$ сопоставлены различные элементы множества $B$ и каждый элемент множества $B$ оказывается сопоставлен некоторому элементу множества $A$ (tool: N).
  \item - Cell: $\mathbb{Z}$ -- множество целых чисел.
  \item - Cell: $\mathbb{Q}$ -- множество рациональных чисел.
  \item - Cell: $\mathbb{I}$ -- множество иррациональных чисел.
  \item - Cell: $\mathbb{R}$ -- множество действительных чисел.
  \item - Cell: $\mathbb{C}$ -- множество комплексных чисел.
  \item - Cell: $x$ -- элемент множества $A$: $x \in A$.
  \item - Cell: $x$ -- не является элементом множества $A$: $x \notin A$.
  \item - Cell: $\forall x \left ( x \in A \Rightarrow x \in B \right ) = A \subset B$.
  \item - Cell: If $A \subset B$ and $B \subset A$, so $A = B$.
  \item - Cell: $\varnothing = \{ \}$, $\forall A: \varnothing \in A$.
\end{enumerate}

\section{Set of Cells: Система множеств}

\begin{enumerate}
  \item - Cell: Взаимно однозначное соответствие -- каждому элементу множества $A$ сопоставлен один и только один элемент множества $B$, так что различным элементам множества $A$ сопоставлены различные элементы множества $B$ и каждый элемент множества $B$ оказывается сопоставлен некоторому элементу множества $A$ (tool: N).
  \item - Cell: $\mathbb{Z}$ -- множество целых чисел.
  \item - Cell: $\mathbb{Q}$ -- множество рациональных чисел.
  \item - Cell: $\mathbb{I}$ -- множество иррациональных чисел.
  \item - Cell: $\mathbb{R}$ -- множество действительных чисел.
  \item - Cell: $\mathbb{C}$ -- множество комплексных чисел.
  \item - Cell: $x$ -- элемент множества $A$: $x \in A$.
  \item - Cell: $x$ -- не является элементом множества $A$: $x \notin A$.
  \item - Cell: $\forall x \left ( x \in A \Rightarrow x \in B \right ) = A \subset B$.
  \item - Cell: If $A \subset B$ and $B \subset A$, so $A = B$.
\end{enumerate}

\section{Set of Cells: Упорядоченные множества}

\begin{enumerate}
  \item - Cell: Взаимно однозначное соответствие -- каждому элементу множества $A$ сопоставлен один и только один элемент множества $B$, так что различным элементам множества $A$ сопоставлены различные элементы множества $B$ и каждый элемент множества $B$ оказывается сопоставлен некоторому элементу множества $A$ (tool: N).
  \item - Cell: $\mathbb{Z}$ -- множество целых чисел.
  \item - Cell: $\mathbb{Q}$ -- множество рациональных чисел.
  \item - Cell: $\mathbb{I}$ -- множество иррациональных чисел.
  \item - Cell: $\mathbb{R}$ -- множество действительных чисел.
  \item - Cell: $\mathbb{C}$ -- множество комплексных чисел.
  \item - Cell: $x$ -- элемент множества $A$: $x \in A$.
  \item - Cell: $x$ -- не является элементом множества $A$: $x \notin A$.
  \item - Cell: $\forall x \left ( x \in A \Rightarrow x \in B \right ) = A \subset B$.
  \item - Cell: If $A \subset B$ and $B \subset A$, so $A = B$.
\end{enumerate}

\section{set of Cells: Размещения и перестановки}

\begin{enumerate}
  \item - Cell: Взаимно однозначное соответствие -- каждому элементу множества $A$ сопоставлен один и только один элемент множества $B$, так что различным элементам множества $A$ сопоставлены различные элементы множества $B$ и каждый элемент множества $B$ оказывается сопоставлен некоторому элементу множества $A$ (tool: N).
  \item - Cell: $\mathbb{Z}$ -- множество целых чисел.
  \item - Cell: $\mathbb{Q}$ -- множество рациональных чисел.
  \item - Cell: $\mathbb{I}$ -- множество иррациональных чисел.
  \item - Cell: $\mathbb{R}$ -- множество действительных чисел.
  \item - Cell: $\mathbb{C}$ -- множество комплексных чисел.
  \item - Cell: $x$ -- элемент множества $A$: $x \in A$.
  \item - Cell: $x$ -- не является элементом множества $A$: $x \notin A$.
  \item - Cell: $\forall x \left ( x \in A \Rightarrow x \in B \right ) = A \subset B$.
  \item - Cell: If $A \subset B$ and $B \subset A$, so $A = B$.
\end{enumerate}

\section{Set of Cells: Сочетания}

\begin{enumerate}
  \item - Cell: Взаимно однозначное соответствие -- каждому элементу множества $A$ сопоставлен один и только один элемент множества $B$, так что различным элементам множества $A$ сопоставлены различные элементы множества $B$ и каждый элемент множества $B$ оказывается сопоставлен некоторому элементу множества $A$ (tool: N).
  \item - Cell: $\mathbb{Z}$ -- множество целых чисел.
  \item - Cell: $\mathbb{Q}$ -- множество рациональных чисел.
  \item - Cell: $\mathbb{I}$ -- множество иррациональных чисел.
  \item - Cell: $\mathbb{R}$ -- множество действительных чисел.
  \item - Cell: $\mathbb{C}$ -- множество комплексных чисел.
  \item - Cell: $x$ -- элемент множества $A$: $x \in A$.
  \item - Cell: $x$ -- не является элементом множества $A$: $x \notin A$.
  \item - Cell: $\forall x \left ( x \in A \Rightarrow x \in B \right ) = A \subset B$.
  \item - Cell: If $A \subset B$ and $B \subset A$, so $A = B$.
\end{enumerate}

\section{Cell} \textbf{Пример 1}

some some

\section{Cell} \textbf{Пример 2}

some some

\section{Cell} \textbf{Analysis}

some some ...

\end{document}
